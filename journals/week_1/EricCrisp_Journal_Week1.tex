\documentclass[11pt,letterpaper]{article}

% Essential packages
\usepackage{fontspec}
\usepackage[utf8]{inputenc}
\usepackage[T1]{fontenc}
\usepackage{geometry}
\usepackage{amsmath}
\usepackage{amsfonts}
\usepackage{amssymb}
\usepackage{graphicx}
\usepackage{booktabs}
\usepackage{array}
\usepackage{longtable}
\usepackage{multirow}
\usepackage{multicol}
\usepackage{float}
\usepackage{caption}
\usepackage{subcaption}
\usepackage{hyperref}
\usepackage{fancyhdr}
\usepackage{datetime}
\usepackage{enumitem}
\usepackage{listings}
\usepackage{xcolor}
\usepackage{verbatim}
\usepackage{url}

% Page setup
\geometry{margin=1in}
\setlength{\parindent}{0pt}
\setlength{\parskip}{6pt}

% Header and footer setup
\pagestyle{fancy}
\fancyhf{}
\lhead{Data Science Capstone Journal}
\rhead{\today}
\cfoot{\thepage}
\renewcommand{\headrulewidth}{0.4pt}

% Code listing setup
\lstset{
    basicstyle=\ttfamily\footnotesize,
    backgroundcolor=\color{gray!10},
    frame=single,
    breaklines=true,
    captionpos=b,
    numbers=left,
    numberstyle=\tiny\color{gray},
    keywordstyle=\color{blue},
    commentstyle=\color{green!60!black},
    stringstyle=\color{red}
}

% Custom commands
\newcommand{\journalentry}[2]{
    \section*{Journal Entry - #1}
    \addcontentsline{toc}{section}{Journal Entry - #1}
    \textbf{Date:} #2 \\
    \textbf{Duration:} 6 hours \\
    \rule{\textwidth}{0.5pt}
}

\newcommand{\objective}[1]{
    \subsection*{Objective}
    #1
}

\newcommand{\activities}[1]{
    \subsection*{Activities Completed}
    #1
}

\newcommand{\findings}[1]{
    \subsection*{Key Findings}
    #1
}

\newcommand{\challenges}[1]{
    \subsection*{Challenges Encountered}
    #1
}

\newcommand{\nextSteps}[1]{
    \subsection*{Next Steps}
    #1
}

\newcommand{\reflections}[1]{
    \subsection*{Reflections}
    #1
}

% Title page setup
\newcommand{\workingtitle}{Hybrid Symbolic-Generative AI Physics-Informed Neural Networks for Generating First-Principle Physics-Based Simulations}
\title{\Large \textbf{DATS 598: Data Science Capstone} \\ 
       \large Journal Entries and Progress Documentation \\
       \vspace{0.5cm}
       \normalsize \textbf{\workingtitle}}
\author{Eric Crisp \\ 
        University of Pennsylvania \\
        \texttt{ecrisp@upenn.edu}}
\date{\today}

\begin{document}

\maketitle
\tableofcontents
\newpage

% ========================================
% SAMPLE JOURNAL ENTRY
% ========================================

\journalentry{Week 1}{\today}

\objective{
To begin the planning and synthesizing part of the project including literature review, data location, and beginning to define scope.}

\activities{
\begin{itemize}
    \item Conducted initial literature review on state of physics-informed neural networks
    \item Began populating database for retrevial of thermodynamic training data using CoolProp/NIST
    \item Set up repositiory, python environment, and docker in case deploying
    \item Developed fundamental tools and data structures and began validating against open source utilities
\end{itemize}
}

\findings{
We want to be able to predict the state of a fluid (i.e., water) when undergoing physical processes like expansion, contraction, and hopefully phase change. To do so, gathering either formula or dataset of points is required. This is tabulated to a SQLite database using grid population from CoolProp.
}

\challenges{
\begin{itemize}
    \item Understanding the technical feasibility of the process
    \item Ensuring the current literature review does not already contain results regarding this process
    \item Understanding and determining data storage, access, and compute limitations
\end{itemize}
}

\nextSteps{
\begin{enumerate}
    \item Finish literature review and summarize
    \item Leverage open source but understand where limitations are likely to occur
    \item Develop timeline and process chart for ensuring progress and hitting milestones
\end{enumerate}
}

\reflections{
While this project is intentionally open ended, define clarity on the goal. We aim to achieve the following:
\begin{enumerate}
    \item Can a neural network architecture, when combined and trained with first-principle physics, develop physically valid equations?
    \item Leveraging LLM, can the physically valid equations be converted into syntactically and mathematically correct code?
    \item Can the process take timeseries data (i.e., sensor data) and derive a simple model (simulation) that captures the neccesary physics?
\end{enumerate}
}

\newpage

% ========================================
% APPENDICES (if needed)
% ========================================

% symbols for directory modeling: ├  └  │  ─  ┬  ┼  ═
\newpage
\appendix
\section{Code Repository Structure}
\begin{verbatim}
pinn_dats598/
├── .gitignore
├── documents/
├── journals/
├── lectures/
├── refs/
├── results/
├── notebooks/
├── src/
│   ├── __init__.py
│   ├── activation.py
│   ├── neural_network.py
│   ├── opensource.py
│   └── perceptron.py
├── tests/
│   ├── __init__.py
│   └── test_perceptron.py
├── data/
└── docs/
\end{verbatim}

\end{document}